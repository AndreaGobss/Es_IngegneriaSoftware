\chapter{CheatSheet}
% Esempio di stili di testo
Questo è un esempio di citazione \cite{attention}. \\[8pt]
Quando usi parole tra virgolette usa:  ``  e '' (due backquote all'inzio e due singoli apici), che vengono più belle. Esempio: ``esempio'' invece di "esempio". \\[8pt]
Alcuni esempi di stili di testo diversi:
\begin{itemize}
    \item \textbf{Grassetto}, \textit{Corsivo}, \underline{Sottolineato}, \texttt{Monospace}
    \item \textsc{Maiuscoletto} e \textsl{Corsivo obliquo}
    \item Testo colorato: \textcolor{red}{rosso}, \textcolor{blue}{blu}, \textcolor{green}{verde}
    \item Questo è un esempio di testo con una nota a piè di pagina.\footnote{Questa è la nota a piè di pagina.}
\end{itemize}




% Elenchi puntati e numerati
Esempio di elenco puntato
\begin{itemize}
    \item Primo elemento
    \item Secondo elemento
    \begin{itemize}
        \item Sotto-elemento
    \end{itemize}
\end{itemize}


Esempio di elenco numerato
\begin{enumerate}
    \item Primo punto
    \item Secondo punto
    \item Terzo punto
\end{enumerate}



% Figura
\begin{figure}[t]
\begin{center}
\includegraphics[scale=0.5]{immagini/logo.png}
\end{center}
\caption{ Esempio di figura}\label{fig:uni}
\end{figure}

Esempio di come la figura \ref{fig:uni} può essere citata nel testo.\\[8pt]

% Esempio di codice
Questo è un esempio di codice usando il pacchetto lstlisting  \ref{lst:python_example}
\begin{lstlisting}[language=Python,  breaklines=true, frame=single, caption={Esempio di codice Python}, label={lst:python_example}]
def greet(name):
    """Stampa un messaggio di benvenuto."""
    print(f"Hello, {name}!")

greet("Nome")
\end{lstlisting}

% Tabella
Questo è un esempio di tabella \ref{tab:esempio}\\[8pt]
\begin{table}[h]
\centering
\begin{tabular}{|c|c|c|}
    \hline
    Colonna 1 & Colonna 2 & Colonna 3 \\
    \hline
    Valore 1 & Valore 2 & Valore 3 \\
    Valore 4 & Valore 5 & Valore 6 \\
    \hline
\end{tabular}
\caption{Esempio di tabella}
\label{tab:esempio}
\end{table}

% Sezioni
Elenco delle possibili sezioni:
\section{Sezione}
\subsection{Sottosezione}
\subsubsection{Sotto-sotto-sezione}